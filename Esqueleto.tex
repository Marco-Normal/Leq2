\documentclass[12pt,openright,twoside,a4paper]{abntex2}
\usepackage[utf8]{inputenc}
\usepackage{times}
\usepackage[T1]{fontenc}
\usepackage{microtype}
\usepackage{amsmath,graphicx,float,amssymb,import,tikz,booktabs,mathtools,indentfirst}
\usepackage[brazilian,hyperpageref]{backref}
\renewcommand{\backrefpagesname}{Citado na(s) página(s):~}
\renewcommand{\backref}{}
\renewcommand*{\backrefalt}[4]{
	\ifcase #1 %
		Nenhuma citação no texto.%
	\or
		Citado na página #2.%
	\else
		Citado #1 vezes nas páginas #2.%
  \fi}
\usepackage[brazil]{babel}
\usepackage[alf]{abntex2cite}
\setlength{\parindent}{1.3cm}
\setlength{\parskip}{0.2cm}
\frenchspacing
\graphicspath{{Imagens/}}
\hypersetup{
    colorlinks=true,
    linkcolor=black,
    filecolor=magenta,
    urlcolor=cyan,
    }
\titulo{\textbf{Determinação de Tempos no Regime Turbulento}}
\autor{Henrique Candido da Silva Ramos, Isadoa pires Gomes, Juan Felipe Cardoso Elias, Marco Túlio Mello Silva, Maria Letícia Teodoro Reis, Vinícius Rocha João Pinheiro}
\local{Brasil}
\data{}
\instituicao{
Universidade de São Paulo- USP
\par
Escola de Engenharia de Lorena}
\tipotrabalho{Relatório técnico}
\preambulo{Relatório apresentado como forma de safadeza}
\begin{document}
\selectlanguage{brazil}
\imprimircapa

\imprimirfolhaderosto*

\begin{resumo}
Tal x e y
\end{resumo}

\begin{simbolos}

\item $\zeta$




\end{simbolos}



\tableofcontents


\textual



\chapter{Introdução}\label{ch: Introdução}



\chapter{Fundamentação Teórica}\label{ch: Fundamentação Teorica}



\chapter{Equipamentos e procedimento experimental}\label{ch: Equipamentos e procedimento experimental}



\chapter{Resultados}\label{ch: Resultados}



\chapter{Conclusão}\label{ch: conclusões}

\end{document}
